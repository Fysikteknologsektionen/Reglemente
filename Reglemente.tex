\documentclass[11pt,a4paper]{article}
\usepackage[utf8]{inputenc}
\usepackage[T1]{fontenc}
\usepackage{lmodern}
\usepackage[dvips]{graphicx}
\usepackage{latexsym}
%\usepackage{here}
\usepackage[swedish]{babel}
\usepackage{fancyhdr} 
\usepackage{lastpage}
\usepackage{amssymb}
\usepackage{moreverb}
\usepackage{textfit}
\usepackage{textcomp}
\usepackage{eurosym}
\usepackage{longtable}
\usepackage{soul}
\usepackage[normalem]{ulem}

\usepackage{CJKutf8}

\fancyhead[R]{Reglemente för Fysikteknologsektionen}
\fancyfoot[C]{\thepage\ (\pageref{LastPage})}
\pagestyle{fancy}
\setlength{\parskip}{8pt}
\setlength{\parindent}{0pt}
\usepackage{enumerate}

\begin{document}
%---------------------------------------------------------
% Titelsidan
%---------------------------------------------------------

\setlength{\headheight}{14pt}

%\setcounter{page}{1}
  \begin{center}
    \textbf{\Huge{Reglemente för}}\\[3mm]
    \textbf{\Huge{Fysikteknologsektionen}}\\
    \vspace{.7 cm}
    \textbf{\Large{Chalmers studentkår}}
    
    
    \vfill
    
    Utarbetad våren 2002\\[5mm]
    Baserad på tidigare stadga och reglemente för Fysikteknologsektionen,
    Maskinteknologsektionens reglemente från 2001, samt synpunkter
    från förtroendevalda vid Fysikteknologsektionen våren 2002.\\[5mm]
    Mattias Johansson\\
    Sektionsordförande 2000/2001\\[5mm]
    Magnus Jonsson\\
    Sektionskassör 2001/2002\\[5mm]
    Carl Sunde\\
    Lekmannarevisor ChS 2001/2002\\[5mm]
    Omarbetad av sektionsstyrelsen våren 2007\\[5mm]
    Uppdaterad av en av sektionsstyrelsen tillsatt arbetsgrupp våren 2012\\[5mm]
    Omarbetad av en av sektionsmötet tillsatt arbetsgrupp hösten 2014 \\[5mm]
    Uppdaterad av sektionsstyrelsen våren 2016 \\
    Uppdaterad av sektionsstyrelsen våren 2017 \\
    Uppdaterad av sektionsstyrelsen hösten 2017 \\
    \vspace{.3 cm}
    \small{Göteborg}\\
    \small{5 oktober 2017}
  \end{center}

\clearpage


\tableofcontents

\clearpage


\section{Allmänt}

\section{Medlemmar}

\subsection{Skyldigheter}
\begin{enumerate}[\thesubsection .1]
\item Medlem som går i första årskursen på programmet Teknisk fysik
     eller Teknisk matematik skall städa sektionslokalen.
 

   \item Medlem som deltagit i kurs bör delta i kursutvärderingen.
\end{enumerate}

\subsection{Ickeoffentliga dokument}

\begin{enumerate}[\thesubsection .1]

  \item Följande dokument har medlem inte rätt att ta del av, såvida det inte behövs för sektionens verksamhet:
  	\begin{itemize}
  		\item[-] Incidenthanteringsprotokoll
	 	\item[-] Dokument innehållande personuppgifter
  	\end{itemize}

\end{enumerate}

\subsection{Hedersmedlemmar}

\subsubsection{Förteckning}

\begin{enumerate}[\thesubsection .1]

  \item Fysikteknologsektionens hedersmedlemmar är:
    \begin{itemize}
      \item Valen Åke som strandade i Träslövsläge.
      \item Schrödingers katt, kanske.\footnote{Under sektionsmötet 2013--05--07 genomfördes en omröstning om nämnda katts medlemskap. Omröstningen skedde enligt mötets önskan genom sluten votering, och rösterna placerades i ett kuvert, som sedan förseglades. Schrödingers katt är därmed \textit{kanske} hedersmedlem.}
    \end{itemize}

\end{enumerate}

\newpage

\section{Inspektor}

\section{Organisation och ansvar}
%%Kapitel 3: Organisation


\newpage

\section{Sektionsmötet}

\subsection{Närvaro-, yttrande- och förslagsrätt}

\begin{enumerate}[\thesubsection .1]

  \item Närvaro- och yttranderätt tillkommer, förutom de som listas i stadgan, även medlem av någon av kårens enheter.

\end{enumerate}


\subsection{Kallelse}

\begin{enumerate}[\thesubsection .1]

\item Kallelse till sektionsmöte  skall innehålla uppgifter om datum, tid och plats, samt preliminär föredragningslista. Denna skall anslås via sektionens officiella kommunikationskanaler.

\item Kallelse till sektionsmöte skall tillsändas sektionsmedlemmar, revisorer, inspektor och kårstyrelsen.

\end{enumerate}

\subsection{Slutgiltig föredragningslista}
\begin{enumerate}[\thesubsection .1]

\item Slutgiltig föredragningslista skall bestå av slutgiltig dagordning, inkomna handlingar såsom revisionsberättelser, motioner, propositioner, rapporter samt nomineringar. Denna skall anslås via sektionens officiella kommunikationskanaler.

\item Slutgiltig föredragningslista till sektionsmöte skall tillsändas sektionsmedlemmar, revisorer, inspektor samt kårstyrelsen.

\end{enumerate}


\subsection{Sammanträdesordning}

\begin{enumerate}[\thesubsection .1]

  \item Sektionsmötet skall på sitt första möte varje verksamhetsår anta en sammanträdesordning på förslag från sektionsstyrelsen.

\end{enumerate}

\subsection{Åligganden}

\begin{enumerate}[\thesubsection .1]

  \item Det åligger sektionsmötet att innan utgången av läsperiod 1
  utöver de i stadgarna definierade åliggandena välja:
    \begin{itemize}
      \item Årskursrepresentant till studienämnden.
    \end{itemize}

  \item Det åligger sektionsmötet att innan utgången av läsperiod 2
    utöver de i stadgarna definierade åliggandena välja:
    \begin{itemize}
      \item Förtroendeposter i FARM.
      \item Förtroendeposter i FnollK.
      \item Oberoende SAMO.
      \item Balnågonting.
    \end{itemize}

  \item Det åligger sektionsmötet att innan utgången av läsperiod 3
    utöver de i stadgarna definierade åliggandena välja:
    \begin{itemize}
      \item Bilnissar.
      \item Blodgrupp.
      \item Fanfareri.
      \item Finform.
      \item FIF.
      \item Kräldjursvårdare.
      \item Sektionsnörd.
      \item Spidera.
      \item Sångförmän.
      \item Tomte och Lucia.
      \item Game Boy
      \item Piff och Puff
    \end{itemize}

  \item Det åligger sektionsmötet att innan utgången av  läsperiod 4
    utöver de i stadgarna definierade åliggandena välja:
    \begin{itemize}
      \item Dragos.
      \item Sektionsstyrelse.
      \item Valberedning.
      \item Studienämnden, förutom årskursrepresentant.
      \item Förtroendeposter i Djungelpatrullen.
      \item Förtroendeposter i F6.
      \item Förtroendeposter i Focumateriet.
       \item Revisorer.
    \end{itemize}

\end{enumerate}

\newpage

\section{Valberedning och personval}

\subsection{Åligganden}

\begin{enumerate}[\thesubsection .1]

  \item Det åligger valberedningen att ansvara för nomineringen till
  följande poster:
    \begin{itemize}
      \item Samtliga poster i sektionsstyrelsen.
      \item Samtliga förtroendeposter.
      \item Övriga ledamöter i Djungelpatrullen.
      \item Övriga ledamöter i F6.
      \item Övriga ledamöter i Focumateriet.
        \item Övriga ledamöter i FARM.
          \item Övriga ledamöter i FnollK. 
        
    \end{itemize}
  \item Det åligger valberedningen att anslå nomineringar till poster i sektionsstyrelsen, förtroendeposter
och övriga medlemmar i sektionskommittéer senast 7 veckodagar före sektionsmöte då relevant inval
äger rum. Nomineringarna skall anslås via någon av sektionens officiella informationskanaler.

\end{enumerate}

\subsection{Val av förtroendeposter}
	\begin{enumerate}[\thesubsection .1]
	\item Förtroendeposter väljs av sektionsmötet.
	\end{enumerate}

\subsection{Val av övriga medlemmar i kommittéer}
	\begin{enumerate}[\thesubsection .1]
	
	\item Övriga medlemmar i kommitté nomineras i grupp av valberedningen. Sektionsstyrelsen
	behandlar godkännande eller avslag av valberedningens gruppnominering på ett styrelsemöte. Detta
	möte skall äga rum tidigast en dag efter att valberedningen anslagit nomineringar och senast samma
	dag som sektionsmötet för inval av förtroendeposter.
	
	\item Vid godkännande av gruppnomineringen betraktas gruppen som preliminärt fastslagen.
	Besked om preliminär fastslagning anslås med nomineringarna omedelbart efter styrelsemötet.
	
	\item Om sektionsmötet sedan väljer in alla förtroendeposter i berörd kommitté i enlighet med
	valberedningens nominering fastslås gruppen omedelbart och de övriga medlemmarna är att betrakta
	som invalda.
	
	\item Om sektionsmötet inte väljer in alla förtroendeposter i berörd kommitté i enlighet med
	valberedningens nominering förkastas den preliminärt fastslagna gruppen. Därefter skall val av
	övriga medlemmar istället företas på sektionsmöte.
	
	\item Vid avslag av gruppnomineringen företas val av övriga medlemmar på sektionsmöte. Eventuellt
	avslag skall grundas i en bedömning av valberedningens arbete eller i uppenbara oriktigheter
	som framkommit till sektionsstyrelsen. Besked om avslag och motivering till det skall anslås med
	nomineringarna omedelbart efter styrelsemöte.

	\end{enumerate}

\newpage

\section{Sektionsstyrelsen}

\subsection{Sammansättning}

\begin{enumerate}[\thesubsection .1]

  \item sektionsstyrelsen består av följande förtroendeposter:
    \begin{itemize}
      \item Sektionsordförande.
      \item Vice sektionsordförande.
      \item Sektionskassör.
      \item Sekreterare.
      \item Skyddsombud.
      \item Informationsansvarig.
      \item Ordförande i SNF.
      \item Ordförande i FnollK.
      \item Ordförande i F6.
      \item Ordförande i FARM.
      \item Ordförande i FOC.
      \item Ordförande i Djungelpatrullen.
    \end{itemize}

\end{enumerate}

\begin{enumerate}[\thesubsection .2]
\item Villkor för ledamöter i sektionsstyrelsen:

\begin{itemize}
\item En kan ej inneha fler än en position i sektionsstyrelsen.
\item Endast kommitténs/nämndens ordförande får vara medlem i sektionsstyrelsen.
\item Revisor, oberoende SAMO, eller ledamot av valberedningen kan ej vara medlem av sektionsstyrelsen.
\item Ledamot av sektionsstyrelsen kan ej inneha annan förtroendepost.
\end{itemize}

\end{enumerate}

\begin{enumerate}[\thesubsection .3]
\item Suppleanter
\begin{itemize}
\item Vice ordförande i respektive kommitté samt studienämnden är suppleant i sektionsstyrelsen.
\item Suppleant övertar ordförandens befogenheter vid styrelsemöten vid ordförandens bortfall, undantaget då styrelsemötet hålls bakom stängda dörrar.
\end{itemize}
\end{enumerate}

\subsection{Styrelsemöten}

\begin{enumerate}[\thesubsection .1]

\item Sektionsstyrelsen skall sammanträda minst 3 gånger per läsperiod.

\item Kallelse till styrelsemöte anslås av sektionsordföranden.

\item Kallelse till styrelsemöte skall senast två dagar innan mötet skickas till ordinarie ledamöter av sektionsstyrelsen, medlem av kommitté och studienämnd, revisorer, oberoende SAMO samt övriga berörda funktionärer.

\item Protokoll från styrelsemöten anslås via sektionens officiella kommunikationskanaler.

\item Ordinarie ledamöter av sektionsstyrelsen har närvaro-, yttrande-, förslags- och rösträtt. 

\item Revisor, oberoende SAMO och ledamöter av valberedningen har närvaro-, yttrande- och förslagsrätt

\item Ledamöter av kommittéer och studienämnd, samt sektionens inspektor, har närvaro- och yttranderätt.

\item Varje medlem har rätt att få fråga behandla på styrelsemöte. Denna ska då skickas till styrelsen senast tre dagar innan mötet.


\item Bakom stängda dörrar

\begin{itemize}
\item Sektionsstyrelsen kan, om synnerliga skäl föreligger, för visst ärende med minst 2/3 majoritet besluta att överläggning sker bakom stängda dörrar.
\item Enbart ordinarie ledamöter av sektionsstyrelsen äger närvarorätt vid möte bakom stängda dörrar.
\item Sektionsstyrelsen kan adjungera in övriga deltagare.
\item Endast beslutsprotokoll fört då mötet hålls bakom stängda dörrar skall anslås.
\item Det som diskuteras bakom stängda dörrar får ej föras vidare till tredje part. 
\end{itemize}

\end{enumerate}





\subsection{Åligganden}

\begin{enumerate}[\thesubsection .1]

  \item Det åligger sektionsstyrelsen:
    \begin{description}
      \item[att] verka för sammanhållningen mellan sektionsmedlemmarna
      och verka för deras gemensamma intressen.
      \item[att] leda sektionens arbete.
      \item[att] verkställa och övervaka genomförandet av sektionsmötesbeslut.
      \item[att] till sektionens medlemmar skicka ut kallelse och slutgiltiga handlingar
      till sektionsmöte i samband med att dessa dokument anslås.
      \item[att] till kårstyrelsen lägga förslag på sektionsavgift. 
      \item[att] framlägga budget till sektionsmötet.
      \item[att] planera Fysikteknologsektionens framtida inriktning och verk\-sam\-het.
      \item[att] fatta beslut i de ärenden som framlägges till sektionsstyrelsen.
      \item[att] varje år tillsammans med studienämnden utse
      sektionens representanter i styrelser och kommittéer inom
      högskolan. Dock krävs för ledamöterna i Programrådet godkännande
      från sektionsmötet.
      \item[att] granska och utse phaddergruppsansvariga under inrådan av
      FnollK.
      \item[att] delta i de av Djungelpatrullen anordnade städdagarna två gånger per
      läsår. Om detta uppfylls får de närvarande gå gratis på nästa
      sektionsaktivafest.
      \item[att] i samråd med de som önskar söka sektionsstyrelsen ta fram en preliminär 
      verksamhetsplan och presentera denna på sektionsmöte innan utgången av läsperiod 4.
\item[att] följa de av sektionen upprättade arbetsordningarna.
    \end{description}

  \item Det åligger sektionsstyrelsens ordförande:
    \begin{description}
      \item[att] tillse att sektionens beslut verkställs.
      \item[att] föra sektionens talan då något annat ej stadgats eller beslutats.
      \item[att] teckna sektionens firma.
      \item[att] tillse att sektionsmöten sammankallas.
      \item[att] leda och övervaka arbetet inom sektionsstyrelsen.
      \item[att] vara Fysikteknologsektionens representant i kårledningsutskottet.
      \item[att] tillse att det finns representanter från sektionsstyrelsen i F:s eller TM:s programråd.
      \item[att] tillsammans med sektionskassören ansvara för Fysikteknologsektionens ekonomi.

    \end{description}
    \noindent
    Sektionsordförande har full insyn i Fysikteknologsektionens alla
    organ och äger rätt att deltaga i deras möten med yttranderätt.

  \item Det åligger sektionsstyrelsens vice ordförande:
    \begin{description}
 
      \item[att] i ordförandes frånvaro överta dennes åligganden.
      \item[att] biträda ordföranden.
      \item[att] i samråd med styrelsen och övriga sektionsaktiva upprätta sektionens verksamhetsberättelse.
    
    \end{description}
  \item Det åligger sektionsstyrelsens  kassör:
    \begin{description}
      \item[att] sköta och ansvara för Fysikteknologsektionens ekonomi tillsammans med ordföranden.
      \item[att] fortlöpande kontrollera kommittéernas räkenskaper och bokför\-ing.
      \item[att] teckna sektionens firma.
      \item[att] genom Chalmers Studentkår uppbära sektionsavgiften.
      \item[att] i samråd med sektionsstyrelsen upprätta preliminärt budgetförslag till första ordinarie höstmötet.
      \item[att] till varje sektionsmöte kunna redogöra för sektionens ekonomiska ställning.
    \end{description}

  \item Det åligger sektionsstyrelsens sekreterare:
    \begin{description}
      \item[att] föra protokoll vid styrelsemöten och senast två läsdagar efter möte överräcka renskrivet protokoll till ordföranden.
      \item[att] tillse att protokoll från såväl styrelse- som sektionsmöten anslås.
      \item[att] tillse att sektionens stadgar, reglemente och förordningar är aktuella och efterlevs.
    \end{description}

  \item Det åligger sektionsstyrelsens skyddsombud:
    \begin{description}
      \item[att] vara ett av sektionens studerandearbetsmiljöombud
      enligt punkt 10.7 samt vara Fys\-ik\-teknolog\-sektionens jämlikhetsansvarige.
    \end{description}
    
  \item Det åligger sektionsstyrelsens informationsansvariga:  
    \begin{description}
      \item[att] sköta kontakt med sektionen samt uppdatera sektionens hemsida. 
      \item[att] tillse att material som inkommer till sektionen anslås eller på annat sätt förmedlas  till den/dem det berör.
      \item[att] vara sektionsstyrelsens representant i Spidera.
    \end{description}    

  \item Det åligger sektionsstyrelsens  övriga ledamöter:
    \begin{description}
      \item[att] bistå sektionsstyrelsen med information.
      \item[att] aktivt deltaga i beslutsprocessen.
      \item[att] redogöra för sin kommittés/studienämnds löpande verksamhet vid styrel\-se\-möten.
      
    \end{description}
\end{enumerate}


\newpage

\section{Nämnder}

\subsection{Studienämnden}

\subsubsection{Villkor för ledamöter i studienämnden}
\begin{itemize}
\item En person kan ej inneha två poster i studienämnden samtidigt
\item Ekonomiskt ansvarig i studienämnden skall vara myndig
\item En person kan ej inneha ordförandepost i en kommitté eller studienämnden samtidigt som denne innehar vice ordförandepost i annan kommitté eller i studienämnden
\end{itemize}

\subsubsection{Sammansättning}

%\begin{enumerate}[\thesubsection .1]
\begin{itemize}

  \item Studienämnden består av följande poster:
    \begin{itemize}

      \item Ordförande
      \item Vice ordförande
      \item Kassör
      \item Sekreterare
      \item Kandidatansvarig
      \item Masteransvarig
      \item Årskursrepresentant åk. 1
      \item Veckobladerist
      \item Ledamot

    \end{itemize}
Där ordförande, vice ordförande och kassör anses vara förtroendevalda.

%\end{enumerate}
\end{itemize}

\subsubsection{Åligganden}

\begin{itemize}%[\thesubsection .1]

  \item Det åligger studienämnden:
    \begin{description}
     
      \item[att]  sammanträda minst tre gånger per läsperiod.
    
      \item[att] ansvara för utvecklandet av utbildningsbevakningen på
				Fysiktekno\-log\-sektionen.
      \item[att] inför Fysikteknologsektionen svara för att F- och TM-teknologernas
      intressen i studie\-frågor och studiemiljö bevakas på ett
      tillfredsställande sätt.
      \item[att] tillse att det finns representant från studienämnden i programråden för F och TM.
      \item[att] på verksamhetsårets första sektionsmöte presentera en verksamhetsplan för det kommande läsåret.   
      \item[att] följa åligganden enligt arbetsbeskrivning.
    \end{description}

	\vspace{5mm}

  \item Det åligger studienämndens ordförande:
    \begin{description}
      \item[att] tillse att studienämndens ålägganden utförs.
      \item[att] leda studienämndens verksamhet.
      \item[att] kalla studienämnden till sammanträde.
      \item[att] tillsammans med kassören ansvara för studienämndens ekonomi.
      \item[att] representera studienämnden i sektionsstyrelsen och vid förhinder se till att suppleant deltager.
      \item[att] i studie- och studiemiljöfrågor representera Fysikteknologsektionen och föra dess talan.
      \item[att] representera F och TM i Utbildningsutskottet, UU, och vid förhinder tillse att suppleant deltager.
      \item[att] inför Fysikteknologsektionen svara för att F- och TM-teknologernas
      intressen i studie\-frågor och studiemiljö bevakas på ett
      tillfredsställande sätt.
    \end{description}

  \item Det åligger studienämndens vice ordförande:
    \begin{description}
      \item[att] assistera ordföranden i dennes åligganden.
      \item[att] ersätta ordföranden när denne inte är närvarande.
      \item[att] ansvara för studiesociala evenemang.
    \end{description}

  \item Det åligger studienämndens kassör:
    \begin{description}
      \item[att] sköta och ansvara för studienämndens ekonomi tillsammans med ordförande.
      \item[att] kontinuerligt föra en granskningsbar redovisning gällande studienämnden:s ekonomi.
      \item[att] mot revisorerna kontinuerligt redovisa den ekonomiska situationen. 
    \end{description}

	\vspace{5mm}
	
    
    \item Det åligger studienämndens sekreterare
     \begin{description}
         \item[att] tillse att protokoll förs på studienämndens möten.
    \item[att] anslå nämndens protokoll enligt stadgarna.
     \end{description}

  \item Det åligger studienämndens medlemmar:
    \begin{description}
      \item[att] vara ordföranden behjälplig.
    \end{description}

\end{itemize}


\newpage

\section{Kommittéer}

\subsection{Villkor för kommittéledamöter}
\begin{itemize}
\item En person kan ej inneha två ledamotsposter i samma kommitté.
\item Ekonomiskt ansvarig i kommitté skall vara myndig
\item En person kan ej inneha ordförandepost i en kommitté eller studienämnden samtidigt som denne innehar viceordförandepost i annan kommitté eller i studienämnden

\end{itemize}

\subsection{Förteckning}

\begin{enumerate}[\thesubsection .1]
  \item Fysikteknologsektionens Sektionskommittéer är:
    \begin{itemize}
      \item Fysikteknologsektionens Arbetsmarknadsgrupp, FARM.
      \item Fysikteknologsektionens Nollningskommitté, FnollK.
      \item Fysikteknologsektionens Sexmästeri, F6.
      \item Fysikteknologsektionens PR-förening och Rustmästeri, Djungelpatrullen.
      \item Fysikteknologsektionens Focumateri, Focumateriet. 
    \end{itemize}
\end{enumerate}

\subsection{Åligganden}

\begin{enumerate}[\thesubsection .1]

  \item Det åligger varje sektionskommitté:
    \begin{description}
    \item[att] på det ordinarie sektionsmöte som följer på det, då kommittén blivit invald, presentera en verksamhetsplan för det kommande verksamhetsåret.     
     \item[att] följa åligganden enligt arbetsbeskrivning. 
    \end{description}

%\end{enumerate}
%\begin{enumerate}[\thesubsection .1]

  \item Det åligger kassören i varje sektionskommitté:
    \begin{description}
      \item[att]  mot revisorerna och sektionskassören kontinuerligt redovisa för den ekonomiska situationen.
       \end{description}
\end{enumerate}

\subsection{FARM}

\begin{enumerate}[\thesubsection .1]

  \item FARM består av följande poster:
    \begin{itemize}
	 \item Ordförande.
         \item Vice Ordförande
	 \item Kassör.
	 \item En ledamot från FnollK.
	 \item 0-5 ledamöter.
   \end{itemize}
Där ordförande, vice ordförande och kassör är förtroendeposter.

\item De övriga ledamöterna nomineras av valberedningen och väljs av sektionsstyrelsen. Valberedningens nomineringar skall tillkännages i nära anslutning till val av förtroendeposter. Nomineringarna av ledamöterna skall diskuteras med sektionsstyrelsen i samband med att nomineringarna av förtroendeposterna offentliggörs. 
 
  \item FARM:s överskott tillfaller Fysikteknologsektionen vid
  verksamhets\-årets slut. Dock skall det finnas ett visst utrymme för
  ersättning i form av representation.

  \item Det åligger FARM:
    \begin{description}
      \item[att] arrangera studiebesök och branschkvällar.   
      \item[att] informera företag om programmen Teknisk fysik samt Teknisk matematik, och deras fördelar.
    \end{description}

  \item Det åligger FARM:s ordförande:
    \begin{description}
      \item[att] tillse att kommitteens åligganden utförs.
      \item[att] leda FARM:s arbete.
      \item[att] tillsammans med kassören ansvara för FARM:s ekonomi.
      \item[att] fungera som kontaktlänk mellan FARM och övriga kommittéer samt företräda FARM i sektionsstyrelsen.
    \end{description}

  \item Det åligger FARM:s vice ordförande:
    \begin{description}
      \item[att] vara kommitténs suppleant i sektionsstyrelsen.
      \item[att] hjälpa FARM-ordföranden i dennes uppgifter så att de utförs på bästa sätt.
    \end{description}

  \item Det åligger FARM:s kassör:
    \begin{description}
      \item[att] tillsammans med ordförande ansvara för och sköta FARM:s ekonomi.
      \item[att] kontinuerligt föra en granskningsbar redovisning gällande FARM:s ekonomi.
    \end{description}

  \item Det åligger övriga FARM-medlemmar:
    \begin{description}
      \item[att] hjälpa FARM- ordföranden i dennes uppgifter så att de utförs på bästa sätt.
    \end{description}

\end{enumerate}


\subsection{FnollK}

\begin{enumerate}[\thesubsection .1]

  \item FnollK består av följande poster:
    \begin{itemize}
      \item Ordförande.
      \item Vice Ordförande
      \item Kassör.
      \item 0-4 ledamöter.
    \end{itemize}

  Där ordförande, vice ordförande och  kassör förtroendeposter.
  

\item De övriga ledamöterna nomineras av valberedningen och väljs av sektionsstyrelsen. Valberedningends nomineringar skall tillkännages i nära anslutning till val av förtroendeposter. Nomineringarna av ledamöterna skall diskuteras med sektionsstyrelsen i samband med att nomineringarna av förtroendeposterna offentliggörs. 

  \item Verksamheten drivs på ideell basis, d.v.s. den ska gå jämnt
  upp. Dock skall det finnas ett visst utrymme för ersättning i form
  av representation.

  \item Det åligger FnollK:
    \begin{description}
      \item[att] genomföra en värdig mottagning i enlighet med kårens och sektionens intentioner.
      \item[att] arrangera aktiviteter för F- och TM-nollan som syftar till
      att införliva Nollan i livet som F- respektive TM-teknolog både vad gäller
      studier och det studiesociala livet.
      \item[att] under ledning av FnollKs  ordförande planera och leda dessa aktiviteter i samråd med berörda organ.
      \item[att] tillse att valen Åke bär nollbricka om gamble så
      tycker.
      \item[att] kanske tillse att Schrödingers katt kanske bär nollbricka.
    \end{description}

  \item Det åligger FnollK:s ordförande:
    \begin{description}
      \item[att] tillse att kommitteens åligganden utförs.
      \item[att] leda och ansvara för FnollKs arbete.
     
     
      \item[att] fungera som kontaktlänk mellan FnollK och övriga kommittéer samt företräda FnollK i sektionsstyrelsen.
      \item[att] tillsammans med FnollKs kassör ansvara för FnollKs ekonomi.
      \item[att] representera Fysikteknologsektionen i Chalmers Studentkårs samarbetsorgan för mottagningen, MoS.
    \end{description}

\item Det åligger FnollK:s vice ordförande:
\begin{description}
 \item[att] vara kommitténs suppleant i sektionsstyrelsen
 \item[att] vid ordförandens frånvaro överta dennes åligganden
\end{description}

  \item Det åligger FnollK:s kassör:
    \begin{description}
      \item[att] sköta och tillsammans med FnollKs ordförande an\-sva\-ra för FnollKs ekonomi.
      \item[att] kontinuerligt föra en granskningsbar redovisning gällande FnollKs  ekonomi.
    \end{description}


  \item Det åligger FnollK:s övriga ledamöter:
    \begin{description}
      \item[att] hjälpa FnollK-ordföranden i dennes uppgifter så att de utförs på bästa sätt.
    \end{description}

\end{enumerate}

\subsection{F6}

\begin{enumerate}[\thesubsection .1]

  \item F6 består av följande poster:
  
    \begin{itemize}
      \item Ordförande, Sexmästare.
      \item Vice ordförande, Sexreterare.
      \item Kassör.
      \item 0-6 ledamöter
    \end{itemize}
Där ordförande, vice ordförande och kassör är förtroendeposter


\item De övriga ledamöterna nomineras av valberedningen och väljs av sektionsstyrelsen. Valberedningends nomineringar skall tillkännages i nära anslutning till val av förtroendeposter. Nomineringarna av ledamöterna skall diskuteras med sektionsstyrelsen i samband med att nomineringarna av förtroendeposterna offentliggörs. 


  \item Verksamheten skall drivas i icke vinstdrivande syfte. Vid
  räkenskaps\-år\-ets slut skall kommitténs tillgångar upp till 0,659
  basbelopp övergå till nästkommande års kommitté. Tillgångar
  därutöver tillfaller sektionen. Visst utrymme för representation får
  förekomma.

  \item Det åligger F6:
    \begin{description}
      \item[att] minst en gång per läsperiod anordna gasque.
      \item[att] ansvara för kalas- och tentamensfestlighetsverksamhet på sektionen.
      \item[att] vara ett komplement till FnollK under mottagningen.
 
    \end{description}

  \item Det åligger F6:s ordförande, Sexmästaren:
    \begin{description}
      \item[att] tillse att kommitténs åligganden utförs.
      \item[att] leda F6:s arbete.
      \item[att] fungera som kontaktlänk mellan F6 och övriga kommittéer samt företräda F6 i sektionsstyrelsen.
      \item[att] att vara Fysiktteknologsektionens representant i
        Gasquerådet, om F6 beslutar att vara medlemmar i
        Gasquerådet.
      \item[att] tillsammans med kassören ansvara för F6:s ekonomi.
    \end{description}

  \item Det åligger F6:s vice ordförande, Sexreteraren:
    \begin{description}
      \item[att] vid sexmästarens frånvaro utföra dennes uppgifter.
      \item[att] vara kommitténs suppleant i sektionsstyrelsen
    \end{description}

  \item Det åligger F6:s kassör:
    \begin{description}
      \item[att] tillsammans med Sexmästaren ansvara för F6:s ekonomi.
      \item[att] kontinuerligt föra en granskningsbar redovisning gällande F6:s ekonomi.
    \end{description}

  \item Det åligger F6:s övriga medlemmar:
    \begin{description}
      \item[att] hjälpa de förtroendevalda i F6:s verksamhet.
    \end{description}

\end{enumerate}

\subsection{Djungelpatrullen}

\begin{enumerate}[\thesubsection .1]

  \item Djungelpatrullens poster utgörs av:
    \begin{itemize}
      \item Ordförande, Överste.
      \item Vice Ordförande, Rustmästare.
      \item Kassör, Skattmästare.
      \item 0--7 adjutanter
    \end{itemize}
Där Ordförande, Vice ordförande och kassör är förtroendeposter

\item De övriga ledamöterna nomineras av valberedningen och väljs av sektionsstyrelsen. Valberedningends nomineringar skall tillkännages i nära anslutning till val av förtroendeposter. Nomineringarna av ledamöterna skall diskuteras med sektionsstyrelsen i samband med att nomineringarna av förtroendeposterna offentliggörs. 


  \item Verksamheten skall drivas i icke vinstdrivande syfte. Vid
  räkenskaps\-årets slut skall kommitténs tillgångar upp till 0,659
  basbelopp övergå till nästkommande års kommitté. Tillgångar
  därutöver tillfaller sektionen. Visst utrymme för representation får
  förekomma.

  \item Det åligger Djungelpatrullen:
    \begin{description}
      \item[att] tillse att sektionshelgonet vördas på ett hedersamt sätt av alla
	Fysikteknologsektionens medlemmar.   
      \item[att] ansvara för att det ordnas arrangemang för medlemmarna på
	Fysikteknologsektionen. Dessa skall hållas i en anda som ökar samman\-håll\-ning\-en på
	Fysikteknologsektionen och ökar kontakten över årskursgränserna.
      \item[att] vara ett komplement till FnollK under mottagningen. 
      \item[att] sköta det löpande underhållet av Fysikteknologsektionens lokaler och egendom.
      \item[att] vårda sektionens traditioner.

   
    \end{description}

  \item Det åligger Djungelpatrullens ordförande, Översten:
    \begin{description}
      \item[att] tillse att kommitténs åligganden utförs. 
      \item[att] leda Djungelpatrullens arbete.
      \item[att] fungera som kontaktlänk mellan Djungelpatrullen och
      övriga kommittéer samt företräda Djungelpatrullen i sektionsstyrelsen.
      \item[att] tillsammans med Skattmästaren ansvara för Djungelpatrullens ek\-o\-nomi.
    \end{description}

  \item Det åligger Djungelpatrullens vice ordförande, Rustmästaren:
    \begin{description}
      \item[att] i överstens frånvaro leda Djungelpatrullens arbete.
      \item[att] leda renoverings- och underhållsarbeten i sektionens lokaler.
      \item[att] ansvara för uthyrning av Focus inventarier.
      \item[att] vara kommitténs suppleant i sektionsstyrelsen.
    \end{description}

  \item Det åligger Djungelpatrullens kassör, Skattmästaren:
    \begin{description}
      \item[att] tillsammans med ordförande ansvara för och sköta Djungelpatrullens ekonomi.
      \item[att] kontinuerligt föra en granskningsbar redovisning gällande Djungelpatrullens ekonomi.
    \end{description}

  \item Det åligger Djungelpatrullens adjutanter:
    \begin{description}
      \item[att] hjälpa de förtroendevalda i DP:s verksamhet.
 
    \end{description}

\end{enumerate}

\subsection{Focumateriet}

\begin{enumerate}[\thesubsection .1]

  \item Focumateriets utgörs av följande förtroendeposter:
  
    \begin{itemize}
      \item Ordförande, kapten.
      \item Vice ordförande, automatpirat.
      \item Kassör, kistväktare.
      \item 0--5 övriga ledamöter
    \end{itemize}
    Där ordförande, vice ordförande och kassör är förtroendeposter.


\item De övriga ledamöterna nomineras av valberedningen och väljs av sektionsstyrelsen. Valberedningends nomineringar skall tillkännages i nära anslutning till val av förtroendeposter. Nomineringarna av ledamöterna skall diskuteras med sektionsstyrelsen i samband med att nomineringarna av förtroendeposterna offentliggörs. 


  \item Verksamheten drivs ideellt, men Focumateriets överskott
  tillfaller Fys\-ik\-teknolog\-sektionen vid bokföringsårets slut. Dock
  skall det finnas ett visst utrymme för ersättning i form av
  representation.

  \item Det åligger Focumateriet:
    \begin{description}
      \item[att] handha Focumaten, samt, efter sektionsstyrelsens bestämmande,
      av sektionen ägda automater samt av sektionen ägd elektronisk
      utrustning.
   
    \end{description}

  \item Det åligger Focumateriets ordförande:
    \begin{description}
      \item[att] leda Focumateriets arbete.
      \item[att] tillse att kommitténs åligganden efterföljs.
      \item[att] tillsammans med kassören ansvara för kommitténs ekonomi.
    \end{description}

 \item Det åligger Focumateriets vice ordförande:
    \begin{description}
      \item[att] vara kommitténs suppleant i sektionsstyrelsen
      
      \item[att] vid ordförandens frånvaro överta dennes åligganden
      
    \end{description}


  \item Det åligger Focumateriets kassör:
    \begin{description}
      \item[att] tillsammans med orföranden ansvara för och sköta Focumateriets ekonomi.
      \item[att] kontinuerligt föra en granskningsbar redovisning gällande focumateriets ekonomi.
    \end{description}

  \item Det åligger Focumateriets övriga ledamöter:
    \begin{description}
      \item[att] hjälpa Focumateriordföranden att efterfölja focumateriets åligg\-and\-en.
    \end{description}

\end{enumerate}

\newpage

\section{Funktionärer}

\subsection{Förteckning}

\begin{enumerate}[\thesubsection .1]

  \item Fysikteknologsektionens sektionsfunktionärer är:
    \begin{itemize}
      \item Fysikteknologsektionens informationsskrift, Finform.
      \item Fysikteknologsektionens Idrottsförening, FIF.
      \item Sångförmännen.
      \item Revisorer.
      \item Studerandearbetsmiljöombud.
      \item Fysikteknologsektionens skyddshelgon Dragos.
      \item Fanfareriet.
      \item Bilnissar.
      \item Blodgruppen.
      \item Kräldjursvårdare.
      \item Dumvästinnehavare.
      \item Tomte och Lucia.
      \item Fysikteknologsektionens webbgrupp, Spidera.
      \item Sektionsnörd.
      \item Balnågonting.
      \item Game boy
      \item Piff och Puff 
    \end{itemize}

\end{enumerate}

\subsection{Åligganden}

\begin{enumerate}[\thesubsection .1]

  \item Det åligger varje sektionsfunktionär:
    \begin{description}
    \item[att] följa upprättad arbetsordning.
  
    \end{description}

\end{enumerate}

\subsection{Finform}

\begin{enumerate}[\thesubsection .1]

  \item Finform är Fysikteknologsektionens informationsskrift och
  skall på ett lättillgängligt sätt presentera intressanta fakta,
  skämt och skvall\-er. 

  \item Finforms redaktion består av chefredaktör tillika ansvarig utgivare, kassör, 2-8
  redak\-tör\-er, samt sektionsfotograf.

  \item Ansvarig utgivare för Finform tillträder efter
  inregistrering enligt gällande lag.

  \item Det åligger Finform-redaktionen:
    \begin{description}
      \item[att] producera minst 4 nummer av Finform per läsår, med för\-del\-ningen 2 på hösten och 2 på våren.
      \item[att] ansvara för tryckning och distribution.

    \end{description}

  \item Det åligger Finforms chefredaktör:
    \begin{description}
       \item[att] vara ansvarig utgivare.
      \item[att] tillse att Finforms åliggande  utförs.
      \item[att] leda Finforms arbete.
    \end{description}


  \item Det åligger Finforms kassör:
    \begin{description}
      \item[att] tillsammans med chefredaktören ansvara för Finforms ekonomi.
      \item[att] ansvara för att Finforms ekonomiska anslag används inom av sektionsstyrelsen fastställd ram.
    \end{description}


  \item Det åligger Finforms ansvariga utgivare:
    \begin{description}
      \item[att] kontrollera Finform så att Finform inte agerar olagligt, krän\-kan\-de eller på annat sätt olämpligt.
      \item[att] Finform agerar på ett lämpligt sätt för att vara Fysikteknolog\-sek\-tionens officiella informationsskrift.
    \end{description}

\end{enumerate}

\subsection{Fysikteknologsektionens Idrottsförening, FIF}
FIF:s syfte är att främja idrottandet på sektionen genom att anordna idrottsaktiviteter och andra evenemang för sektionsmedlemmar.
\begin{enumerate}[\thesubsection .1]

  \item FIF består av ordförande och kassör, samt max 6 övriga
  medlemmar.

  \item Det åligger FIF:s ordförande:
    \begin{description}
      \item[att] tillse att FIF:s åligganden utförs.
      \item[att] fungera som kontaktlänk mellan FIF och andra
      föreningar på sektionen samt sköta kontakten med andra
      sektioners idrotts\-före\-ning\-ar.
      \item[att] tillsammans med kassören ansvara för FIF:s ekonomi.
    \end{description}

  \item Det åligger FIF:s kassör:
    \begin{description}
      \item[att] tillsammans med ordföranden ansvara för FIF:s ekonomi.
      \item[att] sköta kontakten med sektionsstyrelsens kassör.
    \end{description}

\end{enumerate}

\subsection{Fysikteknologsektionens Sångförmän}
Sångförmännens syfte är att förvalta och bevara sektionens sångtraditioner.
\begin{enumerate}[\thesubsection .1]

  \item Sångförmännen är till antalet 0-6.

\end{enumerate}

\subsection{Fysikteknologsektionens Studerandearbetsmiljöombud}

\begin{enumerate}[\thesubsection .1]

  \item Fysikteknologsektionens studerandearbetsmiljöombud (SAMO) är tre till
  antal\-et och består av sektionsstyrelsens skyddsombud samt två oberoende SAMO med
  särskilt ansvar för arbetsmiljön på utbildningsprogrammen Teknisk
  Fysik respektive Teknisk Matematik.
  
  \item Oberoende SAMO är en förtroendepost.
  
\item Oberoende SAMO för Teknisk Matematik skall vara student på nämnda program.

\item Oberoende SAMO för Teknisk Fysik skall vara student på nämnda program.

\item Oberoende SAMO för programmen Teknisk Matematik och Teknisk Fysik får ej vara medlem av sektionskommitté, nämnd på sektionen, valberedningen eller sektionsstyrelsen.

  \item Det åligger studerandearbetsmiljöombuden:
    \begin{description}
      \item[att] tillvarata sektionsmedlemmarnas inressen i
      skyddsfrågor, jäm\-lik\-hets- och jämställdhetsfrågor och samarbeta
      med Chalmers Studentkårs kontaktman samt högskolans skyddsombud.
      \item[att] studerandearbetsmiljöombuden är representerade på
      studienämndens möten då arbetsmiljöfrågor behandlas.
    \end{description}

\end{enumerate}

\subsection{Fysikteknologsektionens skyddshelgon, Dragos}

\begin{enumerate}[\thesubsection .1]
  \item Dragos är sektionens högste beskyddare, och utövar Fanfareriets hög\-sta befäl.
\end{enumerate}


\subsection{Fysikteknologsektionens Fanfareri}
Fanfareriets syfte är att ta hand om sektionens fanor och flaggor.
\begin{enumerate}[\thesubsection .1]

  \item Fanfareriet består av en flaggmarskalk och 1-2 fanbärare.


\end{enumerate}

\subsection{Fysikteknologsektionens Bilnissar}

\begin{enumerate}[\thesubsection .1]

  \item Bilnissarna består av en ekonomisk bilnisse samt en mekanisk
  bilnisse.

  \item Det åligger Bilnissarna:
    \begin{description}
      \item[att] ansvara för de motorfordon som sektionsstyrelsen
      beslutat om, dock endast sådana som sektionen helt eller delvis
      förfogar över.
      \item[att] ansvara för uthyrning av dessa fordon, enligt taxa
      fastställd av sektionsstyrelsen.
    \end{description}

  \item Det åligger ekonomisk bilnisse:
    \begin{description}
      \item[att] vara sektionskassören behjälplig vid ekonomiska
      ärenden rör\-an\-de bilnisse.
    \end{description}

  \item Det åligger mekanisk bilnisse:
    \begin{description}
      \item[att] tillse att ovannämnda fordon underhålls och repareras
      på ett tillfredsställande sätt.
    \end{description}

\end{enumerate}

\subsection{Fysikteknologsektionens Blodgrupp}
Blodgruppens syfte är att uppmuntra sektionsmedlemmarna till att lämna blod
\begin{enumerate}[\thesubsection .1]


  \item Blodgruppen består av en ansvarig och 1-4 ledamöter.

\end{enumerate}

\subsection{Fysikteknologsektionens Kräldjursvårdare}
Kräldjursvårdarens syfte är att ta hand om sektionens slang, Tilde.
\begin{enumerate}[\thesubsection .1]

  \item Det ska finnas en kräldjursvårdare på sektionen.

  \item Sektionen skall inte ha kräldjur på Focus.

\end{enumerate}

\subsection{Fysikteknologsektionens Dumvästinnehavare}

\begin{enumerate}[\thesubsection .1]

  \item Det ska väljas en Dumvästinnehavare på varje sektionsmöte.

  \item Reglemente angående Dumvästens utdelande:
    \begin{enumerate}
      \item Den, som med avsikt att erhålla Dumvästen utfört dumheter
      bör ej vara kvalificerad till denna.
      \item Kriminella handlingar är ej kvalificerade till Dumvästen,
      såvida inte sektionsmötet anser detta.
      \item Om ej tillräckligt kvalificerad dumhet nomineras, kvarstår\\
      Dum\-väst\-en hos innehavaren för tillfället.
    \end{enumerate}

  \item Förteckning över Dumvästinnehavare genom tiderna tillhandahålls av sektionsstyrelsen.

\end{enumerate}

\subsection{Fysikteknologsektionens Tomte och Lucia}
Tomte och Lucias syfte är att främja julstämningen på sektionen.
\begin{enumerate}[\thesubsection .1]

  \item Det ska väljas en Tomte och en Lucia.

\end{enumerate}

\subsection{Spidera}

\begin{enumerate}[\thesubsection .1]

  \item Spidera består av en nätmästare,informationsansvarig från sektionsstyrelsen samt 1--9
  nätmakare. Sektionsmötet väljer 2--10 teknologer till
  Spidera, som i samråd med sektionsstyrelsen utser en nätmästare.

  \item Det åligger Spidera:
    \begin{description}
      \item[att] administrera och utveckla Fysikteknologsektionens internetportal.
    \end{description}

  \item Det åligger Spideras nätmästare:
    \begin{description}
      \item[att] tillse att ansvarig för av Spidera disponerad hårdvara är medlem i Spidera.
      \item[att] fungera som länk mellan sektionsstyrelsen och Spidera.
      \item[att] kalla till möte med Spidera.
    \end{description}


\end{enumerate}

\subsection{Fysikteknologsektionens Sektionsnörd}
Sektionsnördens syfte är att ta hand om allting som rör sektionens prenumeration av Fantomen
\begin{enumerate}[\thesubsection .1]

  \item Det ska väljas en sektionsnörd.

\end{enumerate}

\subsection{Fysikteknologsektionens Balnågonting}
Balnågontings syfte är att anordna bal, middag samt kringaktiviteter som tillhör balen
\begin{enumerate}[\thesubsection .1]

	\item Balnågonting har 0-5 medlemmar samt en representant vardera från FnollK, F6 och Djungelpatrullen.
  
\end{enumerate}

\subsection{Game Boy}
Game Boys syfte är att ta hand om de sällskapsspel som finns på Focus
\begin{enumerate}[\thesubsection .1]

	\item Game Boy består av 0-2 Game Boys
	
\end{enumerate}

\subsection{Piff och Puff}
Piff & Puff är sektionens aktivitetsgrupp, och syftar till att hjälpa sektionen engagera och underhålla fler sektionsmedlemmar, samt hjälpa sektionsmedlemmar med vägledning och tips om hur aktiviteter för sektionsmedlemmar kan genomföras.

\begin{enumerate}[\thesubsection .1]

	\item Piff ch Puff består av 0-4 Piffar
	
\end{enumerate}

\newpage


\section{Intresseföreningar}

\subsection{Förteckning}

\begin{enumerate}[\thesubsection .1]

  \item Fysikteknologsektionens intresseföreningar är:
    \begin{itemize}
    \item F-spexet.
    \item Fysikteknologsektionens Japanförening, \begin{CJK}{UTF8}{min}物理学部日本文化会\end{CJK} .
    \item 3Dteamet
    \item Fysikteknologsektionensspelförening, FyS
    \end{itemize}
\end{enumerate}


\subsection{F-spexet}
F-spexets syfte är att årligen verka för att sätta upp ett spex, och på så sätt sprida spexkulturen i Göteborgsområdet allt medan spexets medlemmar har roligt.


\subsection[Fysikteknologsektionens Japanförening]{Fysikteknologsektionens Japanförening, \begin{CJK}{UTF8}{min}物理学部日本文化会\end{CJK}}
Fysikteknologsektionens Japanförenings syfte är att vara en mötesplats för individer med intresse för japansk kultur och det japanska språket samt att främja och underhålla detta intresse och denna språkkunskap. 

\subsection{3Dteamet}
3Dteamets syfte är att verka för att till sina medlemmar tillgängligöra 3D-skrivare och annan utrustning som föreningen har att tillgå samt att administrera och underhålla utrustning och hemsida.

\subsection{Fysikteknologsektionensspelförening, FyS}
Fysikteknologsektionens spelförenings syfte är att främja intresset av digitala spel på sektionen.

\newpage

\section{Sektionens ekonomi}

\section{Revision och ansvarsfrihet}

\section{Styrdokument}
\subsection{Förteckning}
Följande styrdokument skall finnas på sektionen

\begin{itemize}
\item Sammanträdesordning sektionsmötet
\item Sammanträdesordning styrelsemöte
\item Riktlinjer för verksamhetsberättelse och ansvarsfrihet
\item Arbetsordningar
\end{itemize}

Därtill regleras verksamheten av övriga dokument som sektionsstyrelsen beslutat vara styrdokument. Komplett förteckning över övriga styrdokument skall tillhandahållas av sektionsstyrelsen.

\subsection{Ändrings- och tolkningsfrågor}

\begin{enumerate}[\thesubsection .1]
\item Vad gäller ändring och tolkning av enskilda styrdokument ska detta anges i de individuella dokumenten.

\end{enumerate}

\subsubsection{Tolkningstvister}

\begin{enumerate}[\thesubsection .1]

  \item Vid konflikt med Fysikteknologsektionens övriga styrdokument har reglementet företräde.

\end{enumerate}


\newpage

\section{Sektionens upplösande}
\section{Skyddshelgon}


\section{Fysikteknologsektionens fonder}

\subsection{Förteckning}

\begin{enumerate}[\thesubsection .1]
  \item Fysikteknologsektionens fonder är:
    \begin{itemize}
      \item F-fonden.
	  \item Focus-fonden
    \end{itemize}
\end{enumerate}

\subsection{F-fonden}

\begin{enumerate}[\thesubsection .1]

  \item Sektionens medel bör göras räntebärande genom placering i
  rän\-te\-fond.

  \item F-Fondens medel kan användas till:
    \begin{itemize}
      \item Renovering av Fysikteknologsektionens lokaler.
      \item Införskaffande och renovering av sektionsbil.
      \item Inköp av inventarier.
      \item Att täcka eventuellt överdrag av budgeten.
      \item Sektionens oförutsedda utgifter.
      \item Annat som sektionsmötet finner lämpligt.
    \end{itemize}

  \item Fonden handhas av sektionsstyrelsen.

  \item Uttag ur F-fonden beslutas av sektionsmöte. Sådant uttag kan
  dock ej göras om ärendet inte upptagits på slutlig
  föredragningslista.

  \item Sektionsstyrelsen får för den löpande driften låna medel ur
  F-fonden. Vid sådant lån behöver fonden ej kompenseras för
  ränteförlusten.

  \item Överblivna medel inom Fysikteknologsektionen skall i möjligaste mån
  tillfalla fonden. Har styrelsen vid verksamhetsårets slut ej tagit
  samtliga medel inom budgeten i anspråk, skall överskjutande medel
  avsättas till F-fonden.

  \item F-fondens medel skall göras räntebärande genom placering i
  låg\-risk\-fond\-er. Räntan skall fonderas.

  \item När F-fonden understiger det under året gällande basbeloppet
  skall avsättning till F-fonden ske årligen med minst 15\% av
  inbetalda sektionsavgifter tills fonden återigen uppgår till minst
  ett basbelopp.

\end{enumerate}

\subsection{Focus-fonden}

\begin{enumerate}[\thesubsection .1]

  \item Att fondera medel för framtida upprustning av Focus.

  \item Focus-fondens medel kan användas till:
    \begin{itemize}
      \item Renovering av Focus.
      \item Inköp av inventarier till Focus.
    \end{itemize}

  \item Fonden handhas av sektionsstyrelsen i samarbete med DP.

  \item Uttag ur fonden beslutas av sektionsstyrelsen.

  \item Överblivna medel inom budgetposten Focus upprustning tillfaller Focus-fonden vid verksamhetsårets slut.
  
  \item Samtliga hyresintäkter för uthyrning av Focus tillfaller Focus-fonden.

  \item Focus-fondens medel skall göras räntebärande genom placering i
  låg\-risk\-fond\-er. Räntan skall fonderas.

\end{enumerate}

\section{Kommittéoverall}

\subsection{Utseende}

\begin{enumerate}[\thesubsection .1]
  \item Kommittéoverallen ska:
    \begin{itemize}
      \item Vara svart.
      \item Ha sektionsmärket högt upp på vänster ärm.
      \item Ha Chalmersmärket högt upp på höger ärm.
      \item Ha teknologens namn över bröstficken och/eller på benen.
      \item Ha kommittésoverallsmärket tryckt på ryggen.
    \end{itemize}
\end{enumerate}

\subsection{Bärande}

\begin{enumerate}[\thesubsection .1]

  \item Kommittéoverallen ska bäras på ett sätt som främjar
  samman\-håll\-ning\-en på sektionen och bland övriga studenter.

\end{enumerate}
\begin{comment}
\newpage

\appendix

\section{Förteckning över Dumvästinnehavare}

\label{dumvast}

Här följer samtliga dumvästinnehavare sedan denna förnämliga post
infördes. Att ingen ny och bättre dumhet hittats på nästa sektionsmöte
indikeras genom att antalet sektionsmöten återfinns inom parantes. Den
färskaste innehavaren återfinns överst.

\begin{longtable}{p{55mm}lp{60mm}}
  \emph{Namn} & \emph{Årskurs} & \emph{Dumhet}\\ \hline

Oskar ''Gurkan'' Sjökvist & f14 & Skrev ut ett labb-PM till expfysen i A0-format. 3 gånger. 

Johanna ''Nioreh'' Renman & f14& Hade sönder Tvåan och hamnade på Hvitfeldska med 11 flak öl under en föräldrakväll. \\\hline

Felix ''Felkan'' Eriksson & f13 & Trodde att pasta växer på träd. \\\hline

Per Ljung & f13 & Klagade över att han inte kunde få tag i någon studentlägenhet närmre Chalmers. Det uppdagades sedan att han på Boplats var registrerad som att han gick i grundskolan. När han väl fick en lägenhet sade han upp sin gamla innnan visningen, varpå han nästan missade att tacka ja till den nya.  \\\hline

Sofia ''Sol'' Toivonen & tm12 & Försökte förgäves rensa vasken på Focus med händerna, utan att inse att problemet var att proppen satt i. \\\hline

Holger ''JudÅ'' Lindström & tm12 & Släppte in Per i DP, iklädd overall, gratis på Gasquen när han sa att han var Nollan. Dessutom hetsade han Liam ''Alfred'' att hoppa på en nollbricka av sten, ett hopp som slutade på Sahlgrenska.\\ \hline
  
  Ellinor ''Joel6'' Rånge & tm09 & För att misstagit Äckel för Västtrafik, samt för att ha råkat säga ``då kan ju då gå hem nu'' till en Faktumförsäljare efter att ha köpt dennes sista Faktum.\\ \hline
  
  Tobias ''TJ'' Johansson & f12 & Satt uppe hela natten för att färdigställa en förstudie i expfysen. Dagen därpå bytte både
  han och hans labbpartner till TM.\\ \hline

  Hedvig Sundelin & tm12 & Hamnade på väg till Lappfejden på fel tåg och strandade i Östersund med kläder och en Delta, men utan skor och plånbok.\\ \hline
  
  Christoffer ''Hashe'' Hansson &  f13 & Skrev GU:s intromattedugga istället för Chalmers dito trots guidning av FnollK, text på tavla och tes samt att duggan inte liknade de exempelduggor som delats ut.\\ \hline
    
Ragnar ''Ragge'' Englund &  f12 & För att misstagit M:s sektionsmöte för en lunchföreläsning och 
iklädd overall skänkt pengar till dem. \\ \hline

Annika ''Ankan'' Johansson & tm10 & För att ha motions-hetsat till LP2 mötet så att det höll 
på halva natten.  \\ \hline

Filip ''Fanta'' Hjort & f11 & Ställde  iordning NollK-rummet. Det var även problem med att stänga dörren, men efter att ha blivit övertygad av DP att dom bara hade pimpat dörren lite gick Fanta hem. Det visade sig dock att dom bytt ut dörren mot städskrubbsdörren...\\ \hline

Oscar ''\emph{killen som inte fick upp dörren}'' Kalldal& f11 & För att han inte fick upp dörren när han skulle lämna mötessalen. \\ \hline

F6 11/12 &  & När DP arrangerade julbord innan jul råkade brandlarmet gå. Det visade sig dock att det var F6 som utlöste detta när de mikrade choklad vilket var F6 andra gång och Christmas tredje gång att dra brandlarmet i Forskarhuset. \\ \hline

Rickard Andersson & f10 & Beställde alkohol till pubrundan, kände sedan på dagen till pubrundan att han inte orkade hämta allting själv, så han tar med sig aspar. Vad han inte kontrollerar är om alla aspar fyllt 20. Det var dumt. \\ \hline


Agrin Hilmkil & f10 & Kokar 2 liter linssoppa, orkar inte diska matlåda, äter 2 liter linssoppa, får
våldsamt ont i magen. \\ \hline

Marina Yudanov & f10 & Nominerade Djungelpatrullen för att de gjort
sina maillistor publika, men insåg allt för sent att hon i princip
nominerat sig själv (då hon några minuter tidigare valts in som
adjutant). \\ \hline

  Anton Dalsmo & f07 & Kastade sin ''okrossbara'' mobiltelefon på
  spårvagnsrälsen, varfter en spårvagn nästan körde över den. \\ \hline
  
Soheil Bashirinia och Dündar Göç & f08 & Har i egenskap av Tomte och Lucia
misslyckats med ett av de mycket få åligganden de har genom att vara i
Lund och/eller hemma i sängen under sektionsmötet. \\ \hline

Olof Elias & tm09 & Lyckades sätta in 36 000kr av DPs inkomster från
pubrundan på fel bank. \\ \hline

Kristina Berndtsson & f07 & Betalade sin bostadshyra till
kåren. \\ \hline

Henning Holmgren & f08 & Lät Carl Södersten, mer känd som Hoff, tvätta
åt honom. Resultatet blev dock inte bra. Stora fläckar uppenbarade sig
på kläderna. Efter någon månad visar det sig att fläckarna beror på att
tvättmedlet har en bild på ett skinande glas på framsidan och alltså inte
var tvättmedel, utan diskmedel.\\ \hline

666-nollorna & f09 & Misslyckades med att få tag i alkohol och lokal
till phadderomsitsen inte bara en utan två gånger. Efter sittningen
glömde man dessutom kvar mäsk i Focus, som, om det inte varit för en
påpasslig phadder, hade upptäckts under skyddsronden några timmar senare.\\ \hline

Anton 'Smörgås' Guldberg & f08 & Hyrde ut sin lägenhet i sommar och
ombad hyresgästen att när denne flyttade ut skicka nycklarna till Antons
adress i Stockholm. Vad Anton glömde bort var att han eftersände all
post från Stockholm till Göteborg. Något som ledde till att hans nycklar
till lägenheten i Göteborg så småningom låg i brevinkastet innanför
ytterdörren till lägenheten i Göteborg.\\ \hline

FnollK 2009 &  & Har lagt ungefär tre timmar på att spola undertråd för hand istället för att använda symaskin.\\ \hline

  Andreas Källberg & tm08 & Stoppade en metallpenna i ett hål vid bänkbelysningen i Focus kök så att allt ljus i Focus försvann och det blev svart kring hålet.\\ \hline

  Anton Guldberg (2) & f08 & Bor på Olofshöjd och åkte till skolan via Korsvägen i två veckor innan han upptäckte hur nära skolan han bor. Har dessutom uppgett adressen till SGS-områdeskontor som sin hemadress då han bl.a. tecknade hemförsäkring. \\ \hline

  Mia Romare & f07 & Hon, när hon städade DP-rummet, lyckades välta originaldokumentet över DP:s skapande tillsammans med ett flertal krönikor ner i skurvattnet på golvet.\\ \hline
  
  Marcus Elmer & f04 & Skrev datas tenta i Digital och Datorteknik istället för den för Z och F. Han hade dessutom lyckats sätta Focumama på freeplay och gått därifrån.\\ \hline
  
  Erik Larsson & f06 & Plockade ur hatthyllan på Pedrams (FNollK) brors bil och lade den på taket innan han körde iväg med bilen. Hatthyllan blev totalförstörd då den blåste av i en korsning.\\ \hline
  
  Jonas Preisz & f03 & Åkte vattenskidor och tappade bilnycklarna i sjön. Lyckades senare upprepa bravaden att tappa nycklarna i sjön när han gästade DP-bastun under nollningen.\\ \hline
  
  Marianne Berg & f06 & Lyckades skicka iväg sin dator på tåget hem utan att själv lyckas komma med.\\ \hline

  FnollK 2006 (2) &  & Nollade fel val på muséet. Kaskelotten som strandade i Askim fick istället nollbricka.\\ \hline

  Martin Nyman & f02 & Efter att Britney fått slut på batteri och Martin fått reda på att han var tvungen att köra i minst en timme fick han raskt kärringstopp mitt i en korsning, vilket ledde till bärgning.\\ \hline
  
  Leif Sandqvist & f03 & En hetsig tevekontrollant krävde att Leif skulle betala TV-avgift. Detta gjorde Leif och tänkte sen överklaga, men insåg försent att hans betalning gjorde att han ansågs skyldig. Leif lyckades därmed betala TV-avgift trots att han inte har TV.\\ \hline

  Philip Krantz (2) & f04 & Philip lyckades inför de nya nollorna förklara att \char`\"er första sittning har ni ju nu på torsdag\char`\". D.v.s. den som är hemlig...\\ \hline

  Daniel Westerbeg & f03 & Daniel skulle svinga sig över en bäck och misslyckades. Därefter blev han förvånad och arg över att hans cigaretter var blöta.\\ \hline

  Sektionsmötet feb. 2005 & & Sektionsmötet röstade igenom en motion om att en operation för att ta över alla andra sektioner skulle förberedas. Därefter beslutade mötet att detta var dumt och utsåg därefter sektionsordföranden (som varit emot motionen) som  representant att bära västen.\\ \hline

  Emelie Bylund & f03 & Emelie tyckte att det var ett bra och roligt förslag att elda upp en sko. Hon insåg försent att det var hennes egen sko det rörde sig om.\\ \hline

  Johan Wikstrand & f01 & \\ \hline
  Simon Griph & f01 & \\ \hline
  Viktor Griph & f03 & \\ \hline
  Pauliina Wright & f02 & \\ \hline
  David Kallionemi & f01 & \\ \hline
  Jens Kabo & f00  & \\ \hline
  Karin Magnander & f00 & \\ \hline
  Maria Widmark & f99 & \\ \hline
  Nils Eriksson & f00 & \\ \hline
  Carl Sunde & f96 & \\ \hline
  David Carlen & f98 & \\ \hline
  Linda-Maria Johansson (2) & f99 & \\ \hline
  Gustav Eklund (2) & f99 & \\ \hline
  Carl Bohman (2) & f98 & \\ \hline
  Mattias Henrysson & f93 & \\ \hline
  Henrik S Karlsson & f95 & \\ \hline
  Olof Johansson & f97 & \\ \hline
  Dan Petersson & f95 & \\ \hline
  Maria Johannesson & f96 & \\ \hline
  Fredrik Karlsson & f96 & \\ \hline
  Erik Halvordsson & f96 & \\ \hline
  Mattias Heimdahl & f96 & \\ \hline
  Carl Larsson & f95 & \\ \hline
  Andreas Lindberg & f95 & \\ \hline
  Henrik Karlsson & f95 & \\ \hline
  Erik Setterberg & f95 & \\ \hline
  Kristian Acelsson & & \\ \hline
  Ulf Söderberg & f91 & \\ \hline
  Simon Carlsson & f91 & \\ \hline
  Egil Benito & f91 & \\ \hline
  Jon-Anders Bäcker & kf92 & \\ \hline
  Peter Johansson & f92 & \\ \hline
  Carl Sundblad & f91 & \\ \hline
  Dolph & f92 & \\ \hline
  Morten Postrup & f93 & \\ \hline
  Jesper Nilsson (2) & f92 & \\ \hline
  Martin Gedeck & kf92 & \\ \hline
  Johan Liljequist & f91 & \\ \hline
  Peter Lagerlöf & f91 & \\ \hline
  Jonas Hansen & f90 & \\ \hline

\end{longtable}

\end{comment}
\end{document}